% ========================================================
% PROJECT TEMPLATE, *KNOWLEDGE DISCOVERY*, Summer 2018
% University of Potsdam, by Christoph Schommer, University
% of Luxembourg; 
% ========================================================

\documentclass[11pt]{article}
\usepackage{geometry}
\usepackage{enumitem}
\usepackage{graphicx}
\usepackage[usenames,dvipsnames]{color}

%
\def\MakeMeBlue#1{\textcolor{Blue}{#1}}
\pagenumbering{arabic}
%

\parindent 0pt

%

\title{\MakeMeBlue{Enter the project name here}}
\author{Your Name}
\date{Date of Submission}

% ========================================================
\begin{document}
\maketitle

% ========================================================
\section{Introduction}
This section introduces the field and the task of this project, the motivation behind the task and the choice of methods, a brief description of the experiments and results. \\

An example of using bullet points, in case it is needed.
\begin{itemize}[leftmargin=1cm]
   \item Text 1
   \item Text 2
\end{itemize}

% ========================================================
\section{Related Work}
An example of how to insert references in the article~\cite{hu2011}.


% ========================================================
\section{Methodology}
Introduce the machine learning technique(s) or the algorithm(s) used in the project. \\

An example of inserting figures. Position and width of the figure can be adjusted as needed.
\begin{figure}[!htp]        
  \centering
    \includegraphics[width=0.2\textwidth]{MyProject-KnowledgeDiscovery/image.jpg}
    \caption{Brief description of the figure.}
\end{figure}

% ========================================================
\section{Implementation}

Introduce the data used in the experiments, the setup of the experiments, and the results and/or comparisons. \\

An example of inserting tables. Position of the table can be adjusted as needed.
\begin{table}[!htp] 
  \centering
    \begin{tabular}{| l c r |}   % Alignments can be set here
    \hline
    1 & 2 & 3 \\
    4 & 5 & 6 \\ \hline
    7 & 8 & 9 \\
    \hline
    \end{tabular}
  \caption{Brief description of the table.}
\end{table}

% ========================================================
\section{Conclusion}
Conclude the whole project in short text.

% ========================================================

\begin{thebibliography}{9}

\bibitem{hu2011}
  Hu, Yi and Li, Wenjie,
  \textit{Document sentiment classification by exploring description model of topical terms},
  Computer Speech \& Language,
  25-2, pages 386--403,
  Elsevier, 2011.

\end{thebibliography}
 
\end{document}
