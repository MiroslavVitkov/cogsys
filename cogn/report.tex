\documentclass{article}


\usepackage[backend=bibtex]{biblatex}
\bibliography{report}

\newcommand{\face}[0]{\textit{face }}
\newcommand{\para}[0]{\\ \par \noindent}


\title{}
\author{Miroslav Vitkov}
\date{\today}


\begin{document}


\maketitle


\section{Introduction}
This report describes the manufacturing of a software tool for use in face recognition applications.


\section{Face Recognition by Humans}
The human brain reconises faces in a holistic manner~\cite{sinha}.
This results in robustness against blurring or partial obstruction.
\para
Latency of recognising a face averages 120ms~\cite{sinha}.


\section{Local Binary Patterns}
The algorithm comprises of two steps.
The LBP operator is applied to the input image, resulting in a form of edge detection. 
Then regions are assigned to the resultant image and histograms are computed~\cite{ahonen}.
The algorithms detects negative samples early and skips to the next frame.
It also supports updating of the model without reatraining.


\section{Goal}
The design goal of \face is to be a framework for classifying objects in single camera videos.
Furthermore, human faces recognision should work out of the box and more flexibly than just training a model on a compiled dataset.
Lastly, privacy should be concerned - models should be trainable without a stored database.

\section{IO Capabilities}
\face is able to read from a camera, a video file or a dataset of faces.
The dataset must be organised as follows: one folder per subjects, samples are cropped grayscale faces.
\para
It can write to a video player, video file or a dataset of faces.
\para
Because this is just a thin wrapper over OpenCV, the overhead, compared to using raw OpenCV, should be negleagable.
\para
Because the written dataset comprises of single channel cropped faces in a compressed picture format (JPG), it is extremely space-efficient, 
beaten only by a trained model.


\section{Algorithms}
\face features Local Binary Pattern cascades for both face detection and recognition.
The algorithm is run with it's default parameters - no fine tuning has been carried out.
OpenCV further features the Eigenfaces and Fisherfaces face recognisers.
\para
Labels are stored internally as integers, but provided by the API as strings for convenience.


\section{Design}
The implementation uses the Command design pattern~\cite{schmid} to isolate conceptually different actions the software is able to complete.
This permits the implementaton of new functonalty without modifying old code~\cite{kalemis}.
\para
Read and written datasets are represented as streams of streams.
This approach has the advantages of easy iteration and lazy evaluation.
\para
The proliferal usage of namespaces fasciliates segregation of interfaces.
The lack of statically initialized data guarantees the deterministic order of constructor invocations.


\section{Toolchain}
GNU GCC was the chosen compiler with CMake the meta build system.
The compiler was instructed comply to C++20 specification with extensions turned off.
Thus, the code should build on any compliand compiler.
Features from beyond C++11 were not used.


\section{Artefact}
\face is free~\cite{miro} software featuring the permissive MIT license.
It has been tested on a Debian machine with libopencv-dev package installed, but should work on other UNIX systems well. 


\section{Results}
Training and recognition were tested in a privacy-sensitive environment.
As such, no dataset was recorded but just a model was trained.
Recognition worked significantly faster than the quoted 120ms in humans.
\para
The training sample was 2 minutes of video for 2 subjects.
The recognition accuracy was 94\% \textbf{when} the detector fires, but that result is meaningless for such a small sample.


\section{Possible Applications}
\begin{itemize}
    \item{Record a corpus of faces or other objects from either video frames or a camera over multiple sessions.}
    \item{If training and recognition are concurrently run, get to know new faces in real time.}
    \item{Compare other face recognision algorithms, such as Eigenfaces, Fisherfaces, which require the whole dataset to be in main memory.}
    \item{Additional factor of authentication or surveilance.}
\end{itemize}


\printbibliography


\end{document}
