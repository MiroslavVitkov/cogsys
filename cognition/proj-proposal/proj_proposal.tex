\documentclass{article}
\title{Project Proposal: Face recognition in real time; learning new faces in real time}
\author{'Miroslav Vitkov'}
\begin{document}
\maketitle

\section{Goal}
This is an engineering/scientific project with privacy and security implications.

\subsection{Primary Goals}
One primary goal is to manufacture an autonomous computattional system.
It is to consist of a generic software component and a vaguely specified hardware component.
It must be robust(not crash) and exhibit false positive rate below 1 in 1 hour of continued observation of the same subject.
\par
Another primary goal is to explore which face recognition model or implementation is best suited to continuous updating.
This carries the hallmarks of a classical computer science problem - time and space complexity are top priority.

\subsection{Secondary targets}
\begin{enumerate}
    \item{Acheave reasonable perfomance in low-light conditions - a true positive within the first 2 seconds}
    \item{Demonstrate system stability of at least MTBF=90 days.}
    \item{Be cross-platform and free as in beer and as in speаch.}
    \item{Employ a minimal amount of reasonably popular tools to build and run.}
    \item{Run on an old i3 processor, no dedicated graphics and a 720p camera}
\end{enumerate}

\section{Hypothesys}
I expect the current state of art, Haar features, to be most practical.
This will be tested by fitting it and at least 2 more models to the training data.
Emphasys must be placed both on model selection and model training.

\section{Training Data}
The training data will be collected privately and NOT published.
The product's performance can subsecuently be tested on proliferated datasets, on which it is going to provide pessimistic performance values.

\section{Conclusion}
This project seeks to provide two deliverables.
One is a functional software system.
The other is an objective comparrison of face recognition algorithms in the context of learning new faces realtime.
\end{document}
